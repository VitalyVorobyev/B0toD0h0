\documentclass[a4paper,10pt]{article}
\usepackage[utf8]{inputenc}
\usepackage[english,russian]{babel}
\usepackage{indentfirst}
\usepackage{epsfig}
%\usepackage{mathrsfs}
\usepackage{amsmath}
\usepackage{amssymb}
\usepackage{graphicx}
\usepackage{color}
%\usepackage{floatflt}
%\usepackage{fontenc}
%\usepackage{mathtext}
\usepackage{appendix}


\textwidth = 180mm
\textheight = 240mm
\topmargin = -20mm
\oddsidemargin = -10mm

\RequirePackage{xspace}
\newcommand{\cconj}{\ensuremath{\mathcal{C}}\xspace}
\newcommand{\cpconj}{\ensuremath{\mathcal{CP}}\xspace}
\newcommand{\cptconj}{\ensuremath{\mathcal{CPT}}\xspace}
\newcommand{\dsdpi}{\ensuremath{D^{*+}\to D^0\pi^+}\xspace}

\newcommand{\dkpp}{\ensuremath{D\to K^0_S\pi^+\pi^-}\xspace}
\newcommand{\dnkpp}{\ensuremath{D^0\to K^0_S\pi^+\pi^-}\xspace}
\newcommand{\dbkpp}{\ensuremath{\overline{D}{}^0\to K^0_S\pi^+\pi^-}\xspace}
\newcommand{\bdkp}{\ensuremath{B^0\to D K^+\pi^-}\xspace}
\newcommand{\bdnkp}{\ensuremath{B^0\to D^0 K^+\pi^-}\xspace}
\newcommand{\bdbkp}{\ensuremath{B^0\to \overline{D}{}^0 K^+\pi^-}\xspace}
\newcommand{\bdcpkp}{\ensuremath{B^0\to D_{CP} K^+\pi^-}\xspace}
\newcommand{\bdkstar}{\ensuremath{B^0\to D K^{*0}}\xspace}

\newcommand{\bdk}{\ensuremath{B^+\to D K^+}\xspace}
\newcommand{\bdcpk}{\ensuremath{B^+\to D_{CP} K^+}\xspace}

\newcommand{\dkp}{\ensuremath{D\to K^+\pi^-}\xspace}
\newcommand{\dkpbar}{\ensuremath{D\to K^-\pi^+}\xspace}

\newcommand{\dn}{\ensuremath{D^0}\xspace}
\newcommand{\dnbar}{\ensuremath{\overline{D}{}^0}\xspace}

\newcommand{\ab}{\ensuremath{A_B}\xspace}
\newcommand{\abbar}{\ensuremath{\overline{A}_B}\xspace}

\newcommand{\ad}{\ensuremath{A}\xspace}
\newcommand{\adbar}{\ensuremath{\overline{A}}\xspace}
\newcommand{\dvar}{\ensuremath{(m^2_{+}, m^2_{-})}\xspace}

\newcommand{\aab}{\ensuremath{|A_B|}\xspace}
\newcommand{\aabbar}{\ensuremath{|\overline{A}_B|}\xspace}
\newcommand{\aad}{\ensuremath{|A|}\xspace}
\newcommand{\aadbar}{\ensuremath{|\overline{A}|}\xspace}

\newcommand{\mppsq}{\ensuremath{m^2\left(\pi^+\right)}\xspace}
\newcommand{\mpmsq}{\ensuremath{m^2\left(\pi^-\right)}\xspace}
\newcommand{\mksq}{\ensuremath{m^2\left(K_S^0\right)}\xspace}
\newcommand{\mdsq}{\ensuremath{m^2\left(D^0\right)}\xspace}
\newcommand{\mpsq}{\ensuremath{m_+^2}\xspace}
\newcommand{\mmsq}{\ensuremath{m_-^2}\xspace}
\newcommand{\mmpsq}{\ensuremath{m_{\mp}^2}\xspace}
\newcommand{\ek}{\ensuremath{\varepsilon\left(K_S^0\right)}\xspace}
\newcommand{\eksq}{\ensuremath{\varepsilon^2\left(K_S^0\right)}\xspace}
\newcommand{\epp}{\ensuremath{\varepsilon\left(\pi^+\right)}\xspace}
\newcommand{\eppsq}{\ensuremath{\varepsilon^2\left(\pi^+\right)}\xspace}
\newcommand{\epm}{\ensuremath{\varepsilon\left(\pi^-\right)}\xspace}
\newcommand{\epmsq}{\ensuremath{\varepsilon^2\left(\pi^-\right)}\xspace}
\newcommand{\eppm}{\ensuremath{\varepsilon\left(\pi^{\pm}\right)}\xspace}
\newcommand{\pimodsq}{\ensuremath{\left|{\bf p_i}\right|^2}\xspace}
\newcommand{\pjmodsq}{\ensuremath{\left|{\bf p_j}\right|^2}\xspace}


%\newcommand{\deg}{\ensuremath{^{\circ}}\xspace}


%\title{Model-independent measurement of $D^0$-$\bar D^0$ mixing at a Charm Factory}
\title{Заметки о кинематике Далиц-анализа}
\author{В. Воробьев}
\begin{document}
\maketitle

\section{Вычисление $4$-импульсов через переменные Далица}
\subsection{Постановка задачи}
Рассмотрим трехчастичный распад скалярной частицы на три скалярные частицы (\dnkpp). Амплитуда распада является функцией двух переменных $m_{\pm}^2=m^2_{K_S^0\pi^{\pm}}$. Необходимо предложить вариант вычисления $4$-импульсов всех четырех частиц, если известны \mpsq и \mmsq.

\subsection{Выбор системы отсчета}
Необходимо получить $16$ параметров. Естественные связи следующие: $4$ массы частиц и $4$ связи из закона сохранения энергии-импульса. Остается $16-4-4-2=6$ ''лишних'' степеней свободы. Мы их устраним с помощью следующих ограничений:
\begin{enumerate}
 \item Перейдем в систему покоя $D^0$. Это снимет три степени свободы;
 \item Выберем направление вылета $K_S^0$, совпадающее с осью $z$. После этого остается одна степень свободы --- вращение вокруг оси $z$;
 \item Выберем направление вылета $\pi^+$ (а в силу сохранения импульса, и направление вылета $\pi^-$), лежащее в плоскости $xz$.
\end{enumerate}

Перечисленные условия равносильны следующим уравнениям:
\begin{equation}
\begin{split}
 &p_x(K_S^0) = p_y(K_S^0) = p_y(\pi^{+}) = 0,\\
 &p_i(K_S^0) + p_i(\pi^{+}) + p_i(\pi^{-}) = 0,\quad i = x,y,z.
\end{split}
\end{equation}

Следствиями приведенного выбора кинематических условий также являются соотношения
\begin{equation}
 p_y(\pi^-) = 0,\quad p_x(\pi^+) = -p_x(\pi^-).
\end{equation}

\subsection{Результат}
Последовательно применяя следующие формулы, можно вычислить все компоненты $4$-импульсов:
\begin{equation}
 \begin{split}
  \ek  &= \frac{1}{2m_D}\left(\mpsq + \mmsq - 2m_{\pi}^2\right),\\
  \eppm &= \frac{1}{2m_D}\left(m_D^2 + m_{\pi}^2 - \mmpsq\right),\\
  p_z(K_S^0) & = \sqrt{\eksq-\mksq},\\
  p_z(\pi^{\pm}) &= \frac{1}{2p_z(K_S^0)}\left(m_{\pi}^2+\mksq+2\ek\eppm-m_{\pm}^2\right)\\
  p_x(\pi^+) &= -p_x(\pi^-) = \pm\sqrt{(\ek+\epp)^2-\mpsq - (p^2_z(K_S^0)+p^2_z(\pi^+))^2}.
 \end{split}
\end{equation}

\subsection{Дополнительно}
\begin{equation}
 \mpsq + \mmsq =  2\left(m_{\pi}^2 + m_D\ek\right).
\end{equation}

\section{Масштабирование импульсов}
\subsection{Постановка задачи}
Для того, чтобы переменные Далица \mpsq и \mmsq находились в пределах допустимого фазового пространства, необходимо, чтобы массы дочерних частиц и инвариантная масса $D^0$ находились в табличных значениях. После кинематического фитирования дерева распадов, это условие не всегда реализуется с достаточной точностью. Поэтому рассмотри следующую задачу. Есть три $4$-импульса:
\begin{equation}
 p^0_{K}=(\varepsilon^0_K,{\bf p_K}),\quad p^0_{-}=(\varepsilon^0_+,{\bf p_+}),\quad p^0_{+}=(\varepsilon^0_-,{\bf p_-}).
\end{equation}

Мы хотим с помощью масштабного преобразования
\begin{equation}
 p_{K}=(\varepsilon_K,{\bf p_K}(1+\alpha)),\quad p_{-}=(\varepsilon_+,{\bf p_+}(1+\alpha)),\quad p_{+}=(\varepsilon_-,{\bf p_-}(1+\alpha)),
\end{equation}
где 
\begin{equation}
 \varepsilon_i = \sqrt{m_i^2+\pimodsq(1+\alpha)^2},
\end{equation}
удовлитворить соотношению
\begin{equation}\label{eq:condition}
 m_D^2 = \left(\sum_{i}p_i\right)^2.
\end{equation}

\subsection{Линеаризация}
Предполагая $\alpha\ll1$, получим:
\begin{equation}
 \varepsilon_i \approx \sqrt{m_i^2+\pimodsq}\left(1+\frac{\alpha\pimodsq}{\pimodsq+m_i^2}\right)=\varepsilon^0_i\left(1+\frac{\alpha\pimodsq}{\pimodsq+m_i^2}\right),
\end{equation}
\begin{equation}
 \varepsilon_i\varepsilon_j\approx \varepsilon^0_i\varepsilon^0_j\left[1+\alpha\left(\frac{\pimodsq}{(\varepsilon^0_i)^2}+\frac{\pjmodsq}{(\varepsilon^0_j)^2}\right)\right] \equiv \varepsilon^0_i\varepsilon^0_j\left(1+\alpha\chi_{ij}\right).
\end{equation}

В результате из (\ref{eq:condition}) получаем:
\begin{equation}
 m_D^2-\sum_i m_i^2 \approx 2\sum_{i>j}\left[\varepsilon^0_i\varepsilon^0_j\left(1+\alpha\chi_{ij}\right)-{\bf p_i}{\bf p_j}\left(1+2\alpha\right)\right]
\end{equation}

и

\begin{equation}
 \alpha \approx \frac{m_D^2-\sum_i m_i^2-2\sum_{i<j}\left(\varepsilon^0_i\varepsilon^0_j-{\bf p_i}{\bf p_j}\right)}{2\sum_{i<j}\left(\varepsilon^0_i\varepsilon^0_j\chi_{ij}-2{\bf p_i}{\bf p_j}\right)}=\frac{m_D^2-\left(\sum_ip_i^0\right)^2}{2\sum_{i<j}\left(\pimodsq\frac{\varepsilon^0_j}{\varepsilon^0_i}+\pjmodsq\frac{\varepsilon^0_i}{\varepsilon^0_j}-2{\bf p_i}{\bf p_j}\right)},
\end{equation}



\end{document}

