\documentclass[10 pt,compress,mathserif]{beamer}

%\usepackage[english]{babel}
\usepackage[utf8]{inputenc}
\usepackage{default}

%\usepackage{multirow}
\usepackage{hhline}
%\titlehead{\centering\includegraphics[width=6cm]{pics/B-logo.png}}
\usepackage{color}
\usepackage{colortbl}
\usepackage{array}
\usepackage{amsmath}
\usepackage{amssymb}

\usetheme{Warsaw}

%\usecolortheme{rose}
%\usecolortheme{dolphin}
\usecolortheme{lily}

\RequirePackage{xspace}
\newcommand{\cosbeta}{\ensuremath{\cos{2\varphi_1}}\xspace}
\newcommand{\sinbeta}{\ensuremath{\sin{2\varphi_1}}\xspace}
\newcommand{\cosbetasq}{\ensuremath{\cos^2{2\varphi_1}}\xspace}
\newcommand{\sinbetasq}{\ensuremath{\sin^2{2\varphi_1}}\xspace}
\newcommand{\cpconj}{\ensuremath{\mathcal{CP}}\xspace}
\newcommand{\dkspp}{\ensuremath{D^0\to K_S^0\pi^+\pi^-}\xspace}
\newcommand{\bdh}{\ensuremath{B^0\to \bar D^0h^0}\xspace}
\newcommand{\bdpi}{\ensuremath{B^0\to \bar D^0\pi^0}\xspace}
\newcommand{\bdpp}{\ensuremath{B^0\to \bar D^0\pi^+\pi^-}\xspace}
\newcommand{\bdstpi}{\ensuremath{B^0\to \bar D^{\star0}\pi^0}\xspace}
\newcommand{\bdsth}{\ensuremath{B^0\to \bar D^{\star0}h^0}\xspace}
\newcommand{\bdeta}{\ensuremath{B^0\to \bar D^0\eta}\xspace}
\newcommand{\bdetast}{\ensuremath{B^0\to \bar D^0\eta^{\prime}}\xspace}
\newcommand{\hgg}{\ensuremath{h^0\to \gamma\gamma}\xspace}
\newcommand{\pigg}{\ensuremath{\pi^0\to \gamma\gamma}\xspace}
\newcommand{\etagg}{\ensuremath{\eta\to \gamma\gamma}\xspace}
\newcommand{\etappp}{\ensuremath{\eta\to \pi^+\pi^-\pi^0}\xspace}
\newcommand{\hppp}{\ensuremath{h^0\to \pi^+\pi^-\pi^0}\xspace}
\newcommand{\bdomega}{\ensuremath{B^0\to \bar D^0\omega}\xspace}
\newcommand{\omegappp}{\ensuremath{\omega\to \pi^+\pi^-\pi^0}\xspace}
\newcommand{\aad}{\ensuremath{|f|}\xspace}
\newcommand{\aadbar}{\ensuremath{|\overline{f}|}\xspace}

\defbeamertemplate*{footline}{shadow theme}
{%
  \leavevmode%
  \hbox{\begin{beamercolorbox}[wd=.5\paperwidth,ht=2.5ex,dp=1.125ex,leftskip=.3cm plus1fil,rightskip=.3cm]{author in head/foot}%
    \usebeamerfont{author in head/foot}\insertframenumber\,/\,\inserttotalframenumber\hfill\insertshortauthor
  \end{beamercolorbox}%
  \begin{beamercolorbox}[wd=.5\paperwidth,ht=2.5ex,dp=1.125ex,leftskip=.3cm,rightskip=.3cm plus1fil]{title in head/foot}%
    \usebeamerfont{title in head/foot}\insertshorttitle%
  \end{beamercolorbox}}%
  \vskip0pt%
}

\begin{document}
\title{Measurement of the CKM angle $\varphi_1$ in \bdh, \dkspp decays with the model-independent approach.}
\author{Vitaly Vorobyev}
\institute{Budker Institute of Nuclear Physics,\\
           Novosibirsk State University}
\date{July 2, 2014}
% Создание заглавной страницы
\frame{\titlepage}
% Автоматическая генерация содержания
\frame{\frametitle{Contents}\tableofcontents}

\section{$\Delta t$ resolution}
\begin{frame}
 \frametitle{$\Delta t$ resolution}
 \begin{block}{Detector resolution}
  Is determined by kinematic vertex fit (signal and tagging sides are treated separately).
 \end{block}
 \begin{block}{Non-primary tracks resolution}
  Part of the tag side resolution determined by tracks from secondary vertex.
 \end{block}
 \begin{block}{Kinematic resolution}
  Equation $\Delta t = \Delta z/(\gamma\beta c)$ assumes $B$-mesons are in rest in the CMS system. This approximation leads to distortion of the $\Delta t$ distribution.
 \end{block}
 \begin{block}{Outlier}
  Very wide fraction of the RF.
 \end{block}
\end{frame}


\section{$R_{det}$ parameterization}
\begin{frame}
 \frametitle{Parameterization of the detector resolution in Tatami}
 \begin{block}{Multiple tracks}
  \begin{equation}
   R_{det}^{mult}(\delta z) = G\left(\delta z,0,(s_0+s_1\xi)\sigma_z\right),
  \end{equation}
   where $G$ is Gaussian, $\sigma_z$ is an estimated error from the kinematic fit.
  \end{block}

  \begin{block}{Single track}
   \begin{equation}
    R_{det}^{sing}(\delta z) = (1-f_{tail})G(\delta z,0,s_{main}\sigma_z)+f_{tail}G(\delta z,0,s_{tail}\sigma_z),
   \end{equation}
   where
  \end{block}
\end{frame}


\section{Conclusions}
\begin{frame}
 \frametitle{Conclusions}

\end{frame}

\appendix
\newcounter{finalframe}
\setcounter{finalframe}{\value{framenumber}}

\section{Backup}
\begin{frame}
 \begin{center}
  \begin{LARGE}
   Backup
  \end{LARGE}
 \end{center}
\end{frame}



\setcounter{framenumber}{\value{finalframe}}
\end{document}
