\documentclass[10 pt,compress,mathserif]{beamer}

\usepackage[english,russian]{babel}
\usepackage[utf8]{inputenc}
\usepackage{default}

%\usepackage{multirow}
\usepackage{hhline}
%\titlehead{\centering\includegraphics[width=6cm]{pics/B-logo.png}}
\usepackage{color}
\usepackage{colortbl}
\usepackage{array}
\usepackage{amsmath}
\usepackage{amssymb}

\usetheme{Warsaw}

%\usecolortheme{rose}
%\usecolortheme{dolphin}
\usecolortheme{lily}

\RequirePackage{xspace}
\newcommand{\cosbeta}{\ensuremath{\cos{2\varphi_1}}\xspace}
\newcommand{\sinbeta}{\ensuremath{\sin{2\varphi_1}}\xspace}
\newcommand{\cosbetasq}{\ensuremath{\cos^2{2\varphi_1}}\xspace}
\newcommand{\sinbetasq}{\ensuremath{\sin^2{2\varphi_1}}\xspace}
\newcommand{\cpconj}{\ensuremath{\mathcal{CP}}\xspace}
\newcommand{\dkspp}{\ensuremath{D^0\to K_S^0\pi^+\pi^-}\xspace}
\newcommand{\bdh}{\ensuremath{B^0\to \bar D^0h^0}\xspace}
\newcommand{\bdpi}{\ensuremath{B^0\to \bar D^0\pi^0}\xspace}
\newcommand{\bdpp}{\ensuremath{B^0\to \bar D^0\pi^+\pi^-}\xspace}
\newcommand{\bdstpi}{\ensuremath{B^0\to \bar D^{\star0}\pi^0}\xspace}
\newcommand{\bdsth}{\ensuremath{B^0\to \bar D^{\star0}h^0}\xspace}
\newcommand{\bdeta}{\ensuremath{B^0\to \bar D^0\eta}\xspace}
\newcommand{\bdetast}{\ensuremath{B^0\to \bar D^0\eta^{\prime}}\xspace}
\newcommand{\hgg}{\ensuremath{h^0\to \gamma\gamma}\xspace}
\newcommand{\pigg}{\ensuremath{\pi^0\to \gamma\gamma}\xspace}
\newcommand{\etagg}{\ensuremath{\eta\to \gamma\gamma}\xspace}
\newcommand{\etappp}{\ensuremath{\eta\to \pi^+\pi^-\pi^0}\xspace}
\newcommand{\hppp}{\ensuremath{h^0\to \pi^+\pi^-\pi^0}\xspace}
\newcommand{\bdomega}{\ensuremath{B^0\to \bar D^0\omega}\xspace}
\newcommand{\omegappp}{\ensuremath{\omega\to \pi^+\pi^-\pi^0}\xspace}
\newcommand{\aad}{\ensuremath{|f|}\xspace}
\newcommand{\aadbar}{\ensuremath{|\overline{f}|}\xspace}

\defbeamertemplate*{footline}{shadow theme}
{%
  \leavevmode%
  \hbox{\begin{beamercolorbox}[wd=.5\paperwidth,ht=2.5ex,dp=1.125ex,leftskip=.3cm plus1fil,rightskip=.3cm]{author in head/foot}%
    \usebeamerfont{author in head/foot}\insertframenumber\,/\,\inserttotalframenumber\hfill\insertshortauthor
  \end{beamercolorbox}%
  \begin{beamercolorbox}[wd=.5\paperwidth,ht=2.5ex,dp=1.125ex,leftskip=.3cm,rightskip=.3cm plus1fil]{title in head/foot}%
    \usebeamerfont{title in head/foot}\insertshorttitle%
  \end{beamercolorbox}}%
  \vskip0pt%
}

\begin{document}
\title{Модельно-независимое измерение $CKM$-фазы $\varphi_1$ в распадах \bdh, \dkspp}
\author{Виталий Воробьев,\\
        аспирант 3 года}
\institute{Научный руководитель: А.Е.~Бондарь}
\date{14 мая 2014}
% Создание заглавной страницы
\frame{\titlepage}
% Автоматическая генерация содержания
\frame{\frametitle{Содержание}\tableofcontents}

\section{Введение}
\subsection{Треугольник унитарности}
\begin{frame}
 \frametitle{Треугольник унитарности}
 \begin{small}
 \begin{columns}
 \begin{column}{0.6\textwidth}
  \begin{block}{Матрица $CKM$}
%   \only<1>{
  Связывает с. с. кварков в слабых заряженных токах с массовыми c. с.:
  \begin{equation*}
   \left( \begin{array}{c}
          d^{\prime} \\
          s^{\prime} \\
          b^{\prime} \end{array}
   \right) = 
   \left( \begin{array}{ccc}
          V_{ud} & V_{us} & V_{ub} \\
          V_{cd} & V_{cs} & V_{cb} \\
          V_{td} & V_{ts} & V_{tb}
          \end{array}
   \right)
   \left( \begin{array}{c}
          d \\
          s \\
          b \end{array}
   \right)
  \end{equation*}
  Антикварки преобразуются с помощью сопряженной матрицы.
%   }
%   \only<2>{
%    Параметризация Вольфенштейна $\mathcal{O}(\lambda^4)$:
%    \begin{equation*}
% %   V_{CKM} = 
%    \left( \begin{array}{ccc}
%           1-\frac{\lambda^2}{2}    & \lambda               & A\lambda^3(1-i\eta)\\
%           -\lambda                 & 1-\frac{\lambda^2}{2} & A\lambda^2\\
%           A\lambda^3(1-\rho-i\eta) & -A\lambda^2           & 1
%           \end{array}
%    \right)
%   \end{equation*}
%   $\lambda\approx0.23$, $A\approx 0.8$, $\rho\approx 0.14$, $\eta\approx 0.35$.
%   }
  
  \end{block}
  \includegraphics[width=\textwidth]{pics/UnitarityTriangle.png}
 \end{column}
 \begin{column}{0.4\textwidth}
  \includegraphics[width=\textwidth]{pics/UnitarityTriangleSketch.png}

  \begin{itemize}
   \item Условие унитарности:
  \end{itemize}
  \begin{equation*}
   V_{ud}V^{\star}_{ub}+V_{cd}V^{\star}_{cb}+V_{td}V^{\star}_{tb}=0
  \end{equation*}
 \end{column}
 \end{columns}
 \end{small}
\end{frame}

\subsection{Угол $\beta\equiv\varphi_1$}
\begin{frame}
 \frametitle{Измерение угла $\beta\equiv\varphi_1$: \sinbeta}
 \begin{small}
 \begin{columns}
  \begin{column}{0.5\textwidth}
  \begin{itemize}
   \item Наиболее точные измерения на $B$-фабриках при изучении времени-зависимой \cpconj-асимметрии в переходах $b\to c\bar c s$ (\textcolor{blue}{$B^0\to J/\psi K_S^0$} и др.).
   \item Общее конечное состояние для $B^0$ и $\bar B^0$ позволяет наблюдать интерференцию при осцилляциях $B$-мезонов и обеспечивает чувствительность к фазе амплитуд распадов.
   \item Проблема неоднозначности решения: \textcolor{red}{$2\varphi_1\to (\pi-2\varphi_1)$}.
  \end{itemize}
    %\includegraphics[width=\textwidth]{pics/sin2betaMoriond2014.png}
  \end{column}
  \begin{column}{0.5\textwidth}
   \includegraphics[width=\textwidth]{pics/phi1_ambiguity.png}
  \end{column}
 \end{columns}

 \begin{equation*}
    \mathcal{P}(\Delta t) = \frac{e^{-\Delta t}/\tau_{B^{0}}}{4\tau_{B^0}}\left(1+q_B\left[\mathcal{S}_f\sin(\Delta m_B\Delta t)+\mathcal{A}_f\cos(\Delta m_B\Delta t)\right]\right),
 \end{equation*}
 где $q_B=1 (-1)$ для $B^0 (\bar B^0)$, в Стандартной Модели $\mathcal{S}_f=\pm\sinbeta$, $\mathcal{A}_f=0$.
 \begin{equation*}
  \textcolor{blue}{\sinbeta = 0.682 \pm 0.019}
 \end{equation*}
\end{small}
\end{frame}

\begin{frame}[containsverbatim]
 \frametitle{Измерение угла $\beta\equiv\varphi_1$: \cosbeta}
 \begin{small}

 \begin{columns}
  \begin{column}{0.7\textwidth}
   \begin{itemize}
    \item Наиболее точные измерения выполнены на $B$-фабриках при изучении времени-зависимой \cpconj-асимметрии в переходах $b\to c\bar u d$ с многочастичными распадами $D$ (\textcolor{blue}{$B^0\to [K_S^0\pi^+\pi^-]_{D^0} h^0$, $h^0=\pi^0, \eta, \omega$} и др.). Метод предложен в [1].
 \end{itemize}
  \end{column}
  \begin{column}{0.3\textwidth}
   \includegraphics[width=\textwidth]{pics/DP.png}
  \end{column}
 \end{columns}
  
 \begin{equation*}
 \begin{split}
  &\mathcal{M}\left(m_+^2,m_-^2,\Delta t\right) \propto\\
  &f\left(m_-^2,m_+^2\right)\cos\left(\frac{\Delta m\Delta t}{2}\right)-i\textcolor{blue}{e^{-i2\varphi_1}}\xi_{h^0}(-1)^l f\left(m_+^2,m_-^2\right)\sin\left(\frac{\Delta m\Delta t}{2}\right),
 \end{split}
 \end{equation*}
 где $m_{\pm} \equiv m\left(K_S^0\pi^{\pm}\right)$, $\xi_{h^0}$ --- собственное \cpconj-значение $h^0$, $l$ --- орбитальный момент системы $D^0h^0$, $f\left(m_-^2,m_+^2\right)$ --- амплитуда распада \dkspp.
 
 \begin{columns}
  \begin{column}{0.5\textwidth}
    \begin{itemize}
     \item BaBar {\small [PRL 99, 231802 (2007)]}:
    \end{itemize}
   \begin{center}
    \textcolor{blue}{$\cos2\beta = 0.42 \pm 0.49 \pm 0.09 \pm 0.13$}
   \end{center}
  \end{column}
  \begin{column}{0.5\textwidth}
   \begin{itemize}
    \item Belle {\small [PRL 97, 081801 (2006)]}:
   \end{itemize}
   \begin{center}
    \textcolor{blue}{$\cos2\varphi_1 = 1.87^{+0.40}_{-0.53}{}^{+0.22}_{-0.32}$}
   \end{center}
  \end{column}
 \end{columns}

 \vspace{0.4 cm}
 [1] \verb@Bondar, Gershon, Krokovny, Phys.Lett.B624:1-10,2005@
 \end{small}
\end{frame}

\begin{frame}[containsverbatim]
 \frametitle{Бинированный Далиц-анализ для \dkspp}
 \begin{small}
 \begin{columns}
  \begin{column}{0.7\textwidth}
   \begin{block}{Бинирование диаграммы Далица}
  Диаграмма Далица разбивается на $16$ бинов симметрично относительно замены $m_+^2 \leftrightarrow m_-^2$. Номер бина $i$ принимает значения от $-8$ до $8$, исключая $0$; замена $m_+^2 \leftrightarrow m_-^2$ соответствует инверсии знака $i \leftrightarrow -i$.
 \end{block}
  \end{column}
  \begin{column}{0.3\textwidth}
   \includegraphics[width=\textwidth]{pics/DP_Belle_binning.png}
  \end{column}
 \end{columns}

 \begin{block}{Параметра бинированного анализа}
 \begin{itemize}
  \item \textcolor{blue}{$K_i$} --- вероятность распада флэйворного состояния $D^0$ в бин $i$.
  \item $\delta_D$ --- разность сильных фаз между амплитудами распада $D^0$ и $\bar D^0$:
  \begin{equation*}
  \textcolor{blue}{C_i}=\frac{\int\limits_{\mathcal{D}_i}
            \aad\aadbar
            \cos\delta_D\,d\mathcal{D}
            }{\sqrt{
            \int\limits_{\mathcal{D}_i}\aad^2 d\mathcal{D}
            \int\limits_{\mathcal{D}_i}\aadbar^2 d\mathcal{D}
            }}\,, \quad
  \textcolor{blue}{S_i}=\frac{\int\limits_{\mathcal{D}_i}
            \aad\aadbar
            \sin\delta_D\,d\mathcal{D}
            }{\sqrt{
            \int\limits_{\mathcal{D}_i}\aad^2 d\mathcal{D}
            \int\limits_{\mathcal{D}_i}\aadbar^2 d\mathcal{D}
            }}\,.
  \label{cs}
  \end{equation*}
 \end{itemize}

 \end{block}

  \begin{itemize}
   \item $C_i$ и $S_i$ измерены в когерентных распадах $\psi(3770)\to D^0\bar D^0$ \verb@[CLEO, PRD 82, 112006 (2010)]@.
  \end{itemize}
  \end{small}
\end{frame}

\begin{frame}
 \frametitle{Модельно-независимый подход к \bdh, \dkspp}
 \begin{small}
 Наблюдаемые величины: времени-зависимые вероятности
 \begin{equation*}
 \begin{split}
  N_i\left(\Delta t,\varphi_1\right) &= e^{-\frac{\left|\Delta t\right|}{\tau}}\left(K_{i}+K_{-i}\right)[ 1 + q_{B}\frac{K_{i}-K_{-i}}{K_{i}+K_{-i}}\cos\left(\Delta m\Delta t\right)+\\
  &+q_{B}\xi_{h^0}(-1)^l\frac{\sqrt{K_iK_{-i}}}{K_{i}+K_{-i}}\sin\left(\Delta m\Delta t\right)\left(\textcolor{red}{C_i\sin2\varphi_1+S_i\cos2\varphi_1}\right)]
 \end{split}
 \end{equation*}
 
 \begin{itemize}
  %\item Simultaneous time-dependent fit for all DP bins;
  \item $\sin2\varphi_1$ точно измерен в переходах $b\to c\bar c s$ и может быть зафиксирован.
  \item $K_i$ могут быть точно измерены в распадах $D^{\star\pm}\to D^0\pi^{\pm}$.
  \item Полный интеграл светимости эксперимента Belle в два раза больше использовавшегося в предыдущем, модельно-зависимом, анализе.
 \end{itemize}
 \end{small}
\end{frame}

\subsection{Эксперимент Belle}
\begin{frame}
 \frametitle{Эксперимент Belle}
 \begin{small}
 \begin{columns}
  \begin{column}{0.5\textwidth}
   \includegraphics[width=\textwidth]{pics/KEKB.png}
  \end{column}
  \begin{column}{0.5\textwidth}
   \includegraphics[width=\textwidth]{pics/BelleDetector.png}
  \end{column}
 \end{columns}
  \begin{itemize}
   \item Интеграл светимости на резонансе $\Upsilon(4S)$: $711$ fb$^{-1}$, что соответствует $772\cdot 10^6$ $B\bar B$-парам. Полный интеграл светимости $>1$ ab$^{-1}$.
  \end{itemize}
 \end{small}
\end{frame}

\section{Отбор событий}
\begin{frame}
 \frametitle{Отбор событий}
 \begin{small}
 \begin{columns}
  \begin{column}{0.4\textwidth}
   \begin{block}{Рассмотренные моды}
    \begin{enumerate}
     \item \bdpi;
     \item \bdeta
     \begin{itemize}
      \item \etagg;
      \item \etappp;
     \end{itemize}
     \item \bdomega, \omegappp;
    \end{enumerate}
   \end{block}
   \begin{block}{Кинематические параметры}
     \begin{align*}
     & M_{bc} \equiv \sqrt{\left(E^{cms}_{beam}\right)^2-\left(p^{cms}_{B}\right)^2}\\
     &\Delta E \equiv E^{cms}_{B}-E^{cms}_{beam}
     \end{align*}
   \end{block}

  \end{column}
  \begin{column}{0.6\textwidth}
   \begin{columns}
    \begin{column}{0.5\textwidth}
   \begin{center}
    \includegraphics[width=\textwidth]{pics/de_2dfit_gen_1.png}
   \end{center}
  \end{column}
    \begin{column}{0.5\textwidth}
   \begin{center}
    \includegraphics[width=\textwidth]{pics/mbc_2dfit_gen_1.png}
   \end{center}
  \end{column}
   \end{columns}
   \begin{center}
    $2D$-фит данных моделирования.\\
    Проекция на $\Delta E$ (слева) и на $M_{bc}$ (справа) для моды \bdpi.\\
    \vspace{0.3 cm}
    \textcolor{blue}{$N_{sig}^{fit} = 894 \pm 38$}, \textcolor{blue}{$N_{sig}^{true} = 902$}
   \end{center}
   \end{column}
 \end{columns}
 \end{small}
\end{frame}


\section{Подавление континуума}
\begin{frame}
 \frametitle{Подавление континуума (мода \bdpi)}
 \begin{small}
 \begin{columns}
  \begin{column}{0.4\textwidth}
   \begin{itemize}
    \item Основной фон возникает благодаря случайным комбинациям частиц, рожденных в $q\bar q$-континууме.
    \item Мы разработали процедуру подавления $q\bar q$-фона с помощью инструмена Boosted Decision Trees и получили следующие результаты:
    \begin{itemize}
     \item Доля прошедшего сигнала: \textcolor{blue}{$67.7\%$}
     \item Доля подавленного фона: \textcolor{blue}{$96.9\%$}
    \end{itemize}
   \end{itemize}

  \end{column}
  \begin{column}{0.6\textwidth}
   \begin{center}
    \includegraphics[width=0.5\textwidth]{pics/BackRej_BAM.png}
    \includegraphics[width=0.5\textwidth]{pics/SigSignif_BAM.png}\\
    Подавление фона (слева) и значимость сигнала (справа) как функции доли прошедшего сигнала.\\
    \includegraphics[width=0.5\textwidth]{pics/best_cut_sl.png}\\
    отклик BDTg для \textcolor{blue}{Сигнала} и \textcolor{red}{Континуума}
   \end{center}
  \end{column}
 \end{columns}
 \end{small}
\end{frame}

\section{Toy MC}
\begin{frame}
 \frametitle{Toy MC}
 \begin{small}
 \begin{columns}
  \begin{column}{0.6\textwidth}
  \begin{block}{Условия}
  \begin{itemize}
   \item $5000$ toy экспериментов;
   \item Временное разрешение задавалось одним гауссом с $\frac{\sigma}{\tau_{B^0}}=0.8$;
   \item Доля сигнальных событий: $60.1\%$
   \item Количество сигнальных событий: $840$
   \item Вероятность неверно определить аромат $B$: $0.3$
  \end{itemize}
  \end{block}

  \begin{block}{Результат}
   \begin{itemize}
    \item<1-> Ожидаемая стат. точность для $700$ сигнальных событий:
    \begin{itemize}
     \item $\sigma\left({\cosbeta}\right) = 0.25$
     \item $\sigma\left({\sinbeta}\right) = 0.27$
    \end{itemize}
    \item<2-> Оптимальная доля сигнала: $\approx 60\%$. 
   \end{itemize}
  \end{block}
  \end{column}
  \begin{column}{0.4\textwidth}
    \begin{center}
    \only<1>{
     \cosbeta\\ \includegraphics[width=\textwidth]{pics/mode3_x.png}\\
     \sinbeta\\ \includegraphics[width=\textwidth]{pics/mode3_y.png}
    }
    \only<2>{
     \includegraphics[width=0.8\textwidth]{pics/mode3_rms.png}\\
     \textcolor{blue}{RMS(\cosbeta)} и \textcolor{red}{RMS(\sinbeta)} в зависимости от доли сигнальных событий (согласно порогу BDT).
    }
    \end{center}
  \end{column}
 \end{columns}
 \end{small}
\end{frame}

\section{Выводы}
\begin{frame}
 \frametitle{Выводы}
 \begin{small}

 Выводы:
 \begin{enumerate}
  \item Реконструкция событий \bdh (\hppp и \hgg) готова;
  \item Для моды \bdpi:
   \begin{itemize}
    \item Разработана процедура подавления континуума;
    \item Изучены компоненты фона и реализован $2D$-фит $\Delta E$-$M_{bc}$ распределений.
   \end{itemize}
  \item Простое toy MC показало:
  \begin{itemize}
   \item Метод приблизительно одинаково чувствителен и \sinbeta и \cosbeta. Ожидаемая стат. точность после анализа данных эксперимента Belle составляет $\approx 0.3$;
   \item Оптимальная доля сигнальных событий $\approx 60 \%$.
  \end{itemize}
 \end{enumerate}

 План:
 \begin{itemize}
%  \item Выполнить фит $\tau_{B}$ на полном сигнальном MC;
  \item Выполнить \cpconj-фит на полном сигнальном MC для произвольных значений $\varphi_1$;
  \item Выполнить полный анализ для моды \bdpi;
  \item Добавить к полной процедуре остальные моды;
  \item Изучить систематические эффекты;
  \item ...
 \end{itemize}
\end{small}
\end{frame}

\appendix
\newcounter{finalframe}
\setcounter{finalframe}{\value{framenumber}}

\section{Backup}
\begin{frame}
 \begin{center}
  \begin{LARGE}
   Backup
  \end{LARGE}
 \end{center}
\end{frame}

\begin{frame}
 \frametitle{Motivation}
 \begin{columns}
  \begin{column}{0.5\textwidth}
   \begin{itemize}
    \item To determine sign of the $\cos2\varphi_1$;
    \item Previous Belle measurement is based on $350$ fb${}^{-1}$ [BN 883]. Full data set is twice larger;
    \item Model-independent approach is (probably) preferable for the Belle II data;
    \item This measurement is hardly accessible at LHCb.
   \end{itemize}

  \end{column}
  \begin{column}{0.5\textwidth}
  \begin{block}{\bdpi}
   \includegraphics[width=\textwidth]{pics/B0toD0pi0.png}
  \end{block}
  
  \begin{block}{$\cos2\varphi_1$}
   \begin{itemize}
    \item BaBar {\small [PRL 99, 231802 (2007)]}:
   \end{itemize}
  
   $\cos2\beta = 0.42 \pm 0.49 \pm 0.09 \pm 0.13$
   
   \begin{itemize}
    \item Belle {\small [PRL 97, 081801 (2006)]}:
   \end{itemize}

   $\cos2\varphi_1 = 1.87^{+0.40}_{-0.53}{}^{+0.22}_{-0.32}$
  \end{block}

  \end{column}
 \end{columns}
\end{frame}

\subsection{$\varphi_1$ senitivity}
\begin{frame}
 \frametitle{$\varphi_1$ sensitivity [BN 883]}
 \begin{small}
% I follow BN 883.
 \begin{itemize}
  \item $B$ oscillations:
 \end{itemize}
 \begin{equation}
  \bar B^0\left(\Delta t\right) \propto \bar B^0\cos\left(\frac{\Delta m\Delta t}{2}\right)-ie^{-i2\varphi_1}B^0\sin\left(\frac{\Delta m\Delta t}{2}\right)
 \end{equation}
  \begin{itemize}
  \item $\bar B^0\to D^0h^0$ decay, where $h^0 = \pi^0,\eta,\omega$ (neglecting $D$ oscillations):
 \end{itemize}
 \begin{equation}
  \widetilde{D}^0\left(\Delta t\right) \propto D^0\cos\left(\frac{\Delta m\Delta t}{2}\right)-ie^{-i2\varphi_1}\xi_{h^0}(-1)^l\bar D^0\sin\left(\frac{\Delta m\Delta t}{2}\right),
 \end{equation}
  where $\xi_{h^0}$ is \cpconj eigenvalue of $h^0$, $l$ is orbital moment of the $D^0h^0$ system;
 \begin{itemize}
  \item \dkspp decay (assuming \cpconj conservation in $D$ decay):
 \end{itemize}
 \begin{equation}
 \begin{split}
  &M\left(m_+^2,m_-^2,\Delta t\right) \propto\\
  &f\left(m_-^2,m_+^2\right)\cos\left(\frac{\Delta m\Delta t}{2}\right)-ie^{-i2\varphi_1}\xi_{h^0}(-1)^l f\left(m_+^2,m_-^2\right)\sin\left(\frac{\Delta m\Delta t}{2}\right),
 \end{split}
 \end{equation}
 where $m_{\pm} \equiv m\left(K_S^0\pi^{\pm}\right)$. Let's denote $f \equiv f\left(m_-^2,m_+^2\right)$ and $\bar f \equiv f\left(m_+^2,m_-^2\right)$.
 \end{small}
\end{frame}

\subsection{Events reconstruction}
\begin{frame}[containsverbatim]
 \frametitle{Event reconstruction 1}
 \begin{columns}
  \begin{column}{0.5\textwidth}
   \begin{block}{$\pi^{\pm}$ selection}
  \begin{itemize}
   \item $r\varphi$ SVD hits $>0$;
   \item $rz$ SVD hits $>1$.
  \end{itemize}
 \end{block}
 \begin{block}{$\pi^0$ selection}
  All $\pi^0$ from \verb@Mdst_pi0_Manager@.
 \end{block}
 \begin{block}{$K_S^0$ selection}
 \begin{itemize}
  \item \verb@nisKsFinder@;
  \begin{small}
   \item $0.48\ MeV<m_{K_S^0}<0.52\ MeV$.
  \end{small}
 \end{itemize}
 \end{block}
  \end{column}
  \begin{column}{0.5\textwidth}
   \begin{block}{$D^0$ candidates}
    \begin{itemize}
    \begin{small}
     \item $1.81\ GeV < m_{K_s^0\pi^+\pi^-} < 1.92\ GeV$;
    \end{small}     
     \item $D^0$ decay tree fit with \verb@ExKFitter@ with $K_S^0$ and $D^0$ masses and vertices constrains;
     \item $\chi^2/n.d.f$ of the fit $<1000$.
    \end{itemize}
   \end{block}
  \end{column}
 \end{columns}
\end{frame}

\begin{frame}[containsverbatim]
 \frametitle{Event reconstruction 2}
 \begin{columns}
  \begin{column}{0.5\textwidth}
   \begin{block}{\etagg candidates}
    \begin{itemize}
     \item $\left|m(\gamma\gamma)-m(\eta)\right|<50\ MeV$;
     \item Mass fit with \verb@kfitter@.
    \end{itemize}
   \end{block}
   \begin{block}{\hppp candidates}
    \begin{itemize}
     \item $\left|m(\pi^+\pi^-\pi^0)-m(h^0)\right|<50\ MeV$;
     \item $h^0$ decay tree fit with \verb@ExKFitter@ with $h^0$ and $\pi^0$ masses and $h^0$ vertex constrains.
    \end{itemize}
   \end{block}
  \end{column}
  \begin{column}{0.5\textwidth}
  \begin{block}{\bdh, \hgg}
    \begin{itemize}
     \item \verb@IPTube@ constrained fit with $D^0$ candidate. $h^0$ is not used;
    \end{itemize}
   \end{block}
   \begin{block}{\bdh, \hppp}
    \begin{itemize}
     \item Vertex constrained fit of $D^0$ candidate and two charged $\pi$ from the $h^0$ candidate.
    \end{itemize}
   \end{block}
  \end{column}
 \end{columns}

 \begin{center}
  \begin{equation*}
   \left|\Delta E\right| < 0.3\ GeV,\quad 5.2\ GeV < M_{bc} < 5.3\ GeV.
  \end{equation*}
 \end{center}

\end{frame}

\begin{frame}
 \frametitle{Event reconstruction 3}
 \begin{small}
 Data:
 \begin{columns}
  \begin{column}{0.33\textwidth}
  \begin{center}
   \includegraphics[width=\textwidth]{pics/mk_signal.png}\\
   $m(K_S^0)$\\
  \end{center}
  \end{column}
  \begin{column}{0.33\textwidth}
  \begin{center}
   \includegraphics[width=\textwidth]{pics/md_signal.png}\\
   $m(D^0)$
  \end{center}
  \end{column}
  \begin{column}{0.33\textwidth}
   \begin{center}
    \includegraphics[width=\textwidth]{pics/mpi0_signal.png}\\
   $m(\pi^0)$\\
   \end{center}
  \end{column}
 \end{columns}
 \vspace{0.5 cm}
 Signal MC:
 \begin{columns}
  \begin{column}{0.33\textwidth}
  \begin{center}
   \includegraphics[width=\textwidth]{pics/eta_gg_signal.png}\\
   $m(\etagg)$\\
  \end{center}
  \end{column}
  \begin{column}{0.33\textwidth}
  \begin{center}
   \includegraphics[width=\textwidth]{pics/eta_ppp_signal.png}\\
   $m(\etappp)$\\
  \end{center}
  \end{column}
  \begin{column}{0.33\textwidth}
  \begin{center}
   \includegraphics[width=\textwidth]{pics/omega_signal.png}\\
   $m(\omega^0)$\\
  \end{center}
  \end{column}
 \end{columns}
 \end{small}
\end{frame}

\begin{frame}
 \frametitle{Continuum suppression: Variables}
   \includegraphics[width=0.5\textwidth]{pics/tmva_vars_1.png} \includegraphics[width=0.5\textwidth]{pics/tmva_vars_3.png}\\
   \includegraphics[width=0.5\textwidth]{pics/tmva_vars_2.png} \includegraphics[width=0.5\textwidth]{pics/tmva_vars_4.png}
\end{frame}

\subsection{Multiple candidates}
\begin{frame}
\frametitle{Multiple candidates and signal selection efficiency} 
\begin{small}
\begin{columns}
 \begin{column}{0.78\textwidth}
 \begin{itemize}
%  \item We apply cut \textcolor{blue}{$BDTg>0.98$} which keeps $74.8\%$ of signal and suppress $87.2\%$ of background.
  \item After the BDTg cut we've got $1.046$ candidates per event on the average for the fit region:
  \begin{equation*}
  \begin{split}
   -0.15\ \text{GeV} < &\Delta E < 0.3\ \text{GeV},\\
    5.20\ \text{GeV} < &M_{bc} < 5.29\ \text{GeV}
  \end{split}
  \end{equation*}

  \item But! In many cases, all candidates originate from a single signal decay (e.g. due to random soft $\gamma$'s). Such events do not affect on the signal selection efficiency.
 \end{itemize}
 \end{column}
 \begin{column}{0.22\textwidth}
  \small
   \begin{tabular}{|l|c|} \hline
   \# & events, $\%$ \\ \hline
   $1$ & $95.69$\\ \hline
   $2$ & $4.04$\\ \hline
   $3$ & $0.22$\\ \hline
   $4$ & $0.04$\\ \hline
   $5$ & $<0.01$\\ \hline
   \end{tabular}
  \end{column}
 \end{columns}

 \begin{columns}
  \begin{column}{0.4\textwidth}
   \begin{itemize}
    \item If we consider events containing true background candidates, multiplicity comes to \textcolor{blue}{$1.021$}.
%    \item We choose candidate with the highest $M_{bc}$. It leads to \textcolor{blue}{$0.04\%$} signal loss.
    \item We choose candidate with the smallest $|m_{D}-m_{K_S^0\pi^+\pi^-}|$. It leads to \textcolor{blue}{$0.06\%$} signal loss.
   \end{itemize}

  \end{column}
  \begin{column}{0.6\textwidth}
   \begin{table}[bt]
   \small
   \begin{tabular}{|l|c|c|} \hline
   {Criterion} & {Right decision}, $\%$ & {Efficiency}, $\%$ \\ \hline
   First              & $62.53$ & $99.23$ \\ \hline
   Random             & $50.37$ & $98.99$ \\ \hline
   \textcolor{blue}{$M_{bc}$} & \textcolor{blue}{$80.75$} & \textcolor{blue}{$99.61$} \\ \hline
   \textcolor{blue}{$m_D$}    & \textcolor{blue}{$68.60$} & \textcolor{blue}{$99.36$} \\ \hline
   $\chi^2(D^0)$      & $67.20$ & $99.33$ \\ \hline
   $\left|\cos\theta_{thr}\right|$ & $52.83$ & $99.04$ \\ \hline
   \end{tabular}
   \end{table}
  \end{column}
 \end{columns}
\end{small}
\end{frame}

\begin{frame}
 \frametitle{Elliptic signal region}
 \begin{small}
 \begin{columns}
  \begin{column}{0.33\textwidth}
  \begin{center}
   Signal
   \includegraphics[width=\textwidth]{pics/ellips_sig.png}
  \end{center}
  \end{column}
  \begin{column}{0.33\textwidth}
  \begin{center}
   Background
   \includegraphics[width=\textwidth]{pics/ellips_back.png}
  \end{center}
  \end{column}
  \begin{column}{0.33\textwidth}
  \begin{center}
   Together
   \includegraphics[width=\textwidth]{pics/ellips_gen.png}
  \end{center}
  \end{column}
 \end{columns}

 \begin{table}[bt]
 \small
 \begin{tabular}{|l|c|c|c|c|c|c|} \hline
  Parameter & \multicolumn{2}{c|}{Rectangle}& \multicolumn{2}{c|}{Ellips (equal)} & \multicolumn{2}{c|}{Ellips (inscribed)}\\ \hhline{~------}
            &      Fit   & True &      Fit   & True &  Fit       & True  \\ \hline
$N_S$       &$894\pm38$  &$902$ &$891\pm38$  &$907$ &$855\pm37$  & $867$ \\ \hline
$N_{\rho}$  &$30.2\pm5.6$&$25$  &$27.3\pm5.3$&$26$  &$21.9\pm4.7$& $17$  \\ \hline
$N_{comb}$  &$532 \pm 23$&$543$ &$507\pm23$  &$529$ &$421\pm21$  & $405$ \\ \hline
Purity, $\%$&$61.4\pm2.6$&$61.4$&$62.5\pm2.7$&$62.0$&$65.9\pm2.8$& $67.3$\\ \hline
 \end{tabular}
 \end{table}

 \begin{itemize}
  \item Borders are more or less arbitrary now. To be optimized.
 \end{itemize}
\end{small}
\end{frame}

\subsection{BDT}
\begin{frame}
 \frametitle{Boosted Desicion Trees}
 \begin{columns}
  \begin{column}{0.5\textwidth}
    \only<1-3>{
     \begin{figure}
      \centering
      \includegraphics[width=\textwidth]{pics/DT.png}\\
      \caption{{\it Weak} classifier --- decision tree with a tree depth of about $2$ or $3$, that has very little discrimination power.}
     \end{figure}
    }
    \only<4->{\begin{itemize}
     \item<4-> The boosted event classification:
     \begin{equation}
      y({\bf x}) = \frac{1}{N}\sum_{i=1}^{N}\ln{\alpha_i}h_i({\bf x}),
     \end{equation}
     where $h_i({\bf x})=\pm 1$ is a response of an $i$-th individial classifier.
     Small (large) $y$ values indicate a background-like (signal-like) event.
     \item<5-> Weakness and smallness of individual classifiers make training procedure more robust.
    \end{itemize}
    }

  \end{column}
  \begin{column}{0.5\textwidth}
   \only<1-5>{\begin{itemize}
    \item<1-> {\bf Classification problem}.
    \item<2-> Collection of $N=\mathcal O(10^3)$ {\it weak~classifiers}.
    \item<3-> {\it Adaptive Boost}: events that were misclassified during the trainig of a desicion tree are given a higher event weight $\alpha^{\beta}$ in the thaining of the following tree:
    \begin{equation*}
     \alpha = \frac{1-err}{err},
    \end{equation*}
    where $err<0.5$ is a misclassification rate and $\beta<1$ is a learning rate parameter.
   \end{itemize}
   }
   \only<6>{\includegraphics[width=\textwidth]{pics/tmva_bdt.png}}
  \end{column}
 \end{columns}
\end{frame}

\setcounter{framenumber}{\value{finalframe}}
\end{document}
